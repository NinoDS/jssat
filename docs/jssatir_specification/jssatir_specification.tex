\documentclass[twocolumn]{article}

\usepackage{listings}
\usepackage{color}
\usepackage{amsmath}
\usepackage{array}
\usepackage{fancyhdr}
\usepackage{titling}
\usepackage[hidelinks]{hyperref}
\usepackage{natbib}

\setlength{\parindent}{0pt}

% code
\lstdefinelanguage{jssatir}{
  keywords={fn, return, jump, if, else},
  keywordstyle=\color{blue}\bfseries
}

\lstset{
  language=jssatir,
  % backgroundcolor=\color{lightgray},
  basicstyle=\footnotesize\ttfamily,
  numberstyle=\footnotesize
}

% code macro
\newcommand{\code}{\lstinline[mathescape=true,escapeinside={(*@}{@*)}]}

% \newenvironment{code3}{\begin{lstlisting}}{\end{lstlisting}}

% \newenvironment{code2}%
% [1][]{%
%   \begin{lstlisting}[caption=#1,mathescape=true,escapeinside={(*@}{@*)}]%
% }%
% {\end{lstlisting}}%

% https://tex.stackexchange.com/a/112212
\makeatletter
\DeclareRobustCommand{\rvdots}{%
  \vbox{
    \baselineskip4\p@\lineskiplimit\z@
    \kern-\p@
    \hbox{ }
    \hbox{.}\hbox{.}\hbox{.}
  }}
\makeatother

\title{JSSAT IR Specification}
\author{\textsc{SirJosh}}
\date{\today}

\begin{document}

\maketitle

\begin{abstract}
  JSSAT IR is used to represent dynamic programming languages, such as JavaScript,
  Lua, or Lisp. By introducing a formal specification for JSSAT IR, it makes reasoning
  about the validity of JSSAT IR transformations easier, verifiable, and ensures
  that there is a correct interpretation of JSSAT IR.
\end{abstract}

\section{Values}

In JSSAT IR, registers hold values. The different possibilities for what a value
may be are enumerated in this section. In addition, the memory model of values are
described.

\subsection{Memory Model}

There are two kinds of values: primal, and referential. The characteristics of the
two kinds of values are compared below:

\begin{center}
  \begin{tabular}{ |c|c|c| }
    \hline
               & Primal    & Referential      \\
    \hline
    Comparison & by value  & by heap position \\
    \hline
    Mutability & immutable & mutable          \\
    \hline
    Size       & fixed     & grows            \\
    \hline
  \end{tabular}
\end{center}

\subsubsection{The Heap}

Whenever a referential value gets created, the contents (represented by $v_n$) of
the value are placed at the end of the heap (in this paper, this execution model
of JSSAT is single threaded).

\begin{center}
  \begin{tabular}{ |c|c|c|c| }
    \hline
    $v_0$ & $v_1$ & $v_2$ & \dots \\
    \hline
  \end{tabular}
\end{center}

When a referential value is placed within the heap, the position it's in (the $n$ of $v_n$)
is used for equality comparison.

\subsection{Referential Values}

\subsubsection{Records}

A record is a mapping of values to values, \(\{ V_k \mapsto V_v \}\). Specifc operations
that may be performed on records are listed in the linear instruction section. Notably,
any value may be used as a key of a record.

\subsubsection{Lists}

A list is a list of values, \([ v_0,\, v_1,\, \dots,\, v_n ]\). The functionality
of lists can be replicated using records, but a seperate data structure is used
to model lists, as using a list makes code generation more efficient (this is
because lists will generate list-like behavior found in other languages).

\subsubsection{Tuples}

A tuple is a heterogeneous fixed-size list of values, \((v_0,\, v_1,\, \dots,\, v_n)\).
The functionality of tuples can be replicated using records, but a seperate data
structure is used to model tuples, as using a tuples makes code generation more
efficient (this is because tuples will generate struct-like behavior found in other
languages).

\subsection{Primal Values}

\subsubsection{Atoms}

Atoms are unique values, which are unable to be equal to any other value other than
themselves. This makes them useful for representing unique record keys, among other
things. In this paper, atoms are represented as \textbf{bold text}.

\subsubsection{Numbers}

In dynamic programming language, there is often (JavaScript, Lua) one number type
which is treated as a floating point number and, when possible, treated as an integer
for efficiency. JSSAT exposes a variety of number types and has optimization passes
to convert number types into more efficient number types (such as \code{f64} $\to$ \code{i32})
when provable.

\begin{center}
  \begin{tabular}{ |c|c|c|c|c|c| }
    \hline
    bits:            & 8                & 16                & 32                & 64                & \(\infty\)           \\
    \hline
    floating point   &                  & \text{\code{f16}} & \text{\code{f32}} & \text{\code{f64}} &                      \\
    \hline
    signed integer   & \text{\code{i8}} & \text{\code{i16}} & \text{\code{i32}} & \text{\code{i64}} &                      \\
    \hline
    unsigned integer & \text{\code{u8}} & \text{\code{u16}} & \text{\code{u32}} & \text{\code{u64}} &                      \\
    \hline
    big integer      &                  &                   &                   &                   & \text{\code{bigint}} \\
    \hline
  \end{tabular}
\end{center}

\subsubsection{Function Pointers}

A function pointer is a value which contains a reference to a function. These can
be used to build up polymorphic objects with dynamic dispatch. There are instructions
to call the functions that values point to, which will be detailed later.

\subsubsection{Strings}

A string is a contiguous buffer of bytes (encoding unspecified), currently only
treated as a primal type because JavaScript strings are treated like value types.
Will probably fix this later.

\subsubsection{Booleans}

A boolean is \code{true} or \code{false}, used when determining conditions.

\section{Program Model}

A JSSAT function is in SSA form, uses registers to hold values, and can only perform
divergent control flow as a terminating instruction.

\begin{lstlisting}[caption=An addition function, mathescape=true,escapeinside={(*@}{@*)}]
fn $add$($r_1$, $r_2$)
    $r_3$ $\leftarrow$ $r_1 + r_2$
    return $r_3$
\end{lstlisting}

In the most general form, a function can have any number of arguments, any number
of linear instructions, and a single terminating instruction at the end.

\begin{lstlisting}[mathescape=true,escapeinside={(*@}{@*)}]
fn $f$($r_1$, $\dots$, $r_n$)
    $\rvdots$ linear instructions
    terminating instruction
\end{lstlisting}

As JSSAT functions are in SSA form, once a value is assigned to a register, it cannot
be modified.

\begin{lstlisting}[caption=An invalid function,mathescape=true,escapeinside={(*@}{@*)}]
fn $increment$($r_1$)
    $r_1$ $\leftarrow$ $r_1 + 1$ $\textit{invalid, violates SSA form}$
    return $r_1$
\end{lstlisting}

A JSSAT function can have any number of linear instructions. A linear instruction
is one that does not diverge control flow, meaning that for a given linear instruction
\({instructions}_{{index}}\), the next instruction to execute must be \({instructions}_{{index}+1}\).

\begin{lstlisting}[caption=Linear instructions,mathescape=true,escapeinside={(*@}{@*)}]
fn $f$($r_1$)
    $r_2$ $\leftarrow$ $r_1 + 1$  $\textit{// addition is linear}$
    $r_3$ $\leftarrow$ $f$($r_2$) $\textit{// function calls are linear}$
    return $r_3$
\end{lstlisting}

\subsection{Execution Model}

A JSSAT IR program begins at a function.

We first bind the parameter registers to the values given to the function:

\begin{center}
  Registers \\
  \begin{tabular}{ |c|c|c| }
    \hline
    $r_1$ & $r_2$ & \dots \\
    \hline
    2     & 3     & \dots \\
    \hline
  \end{tabular}
\end{center}

\begin{lstlisting}[mathescape=true,escapeinside={(*@}{@*)}]
fn $add$($r_1$, $r_2$) $\textit{// <-- we are here}$
    $r_3$ $\leftarrow$ $r_1 + r_2$
    if $r_3$   $\rightarrow$ return $r_3$
    else  $\rightarrow$ return $r_3$
\end{lstlisting}

Then, we execute every instruction, binding the results of executing instructions
to registers if necessary.

\begin{center}
  Registers \\
  \begin{tabular}{ |c|c|c|c| }
    \hline
    $r_1$ & $r_2$ & $r_3$ & \dots \\
    \hline
    2     & 3     & 5     & \dots \\
    \hline
  \end{tabular}
\end{center}

\begin{lstlisting}[mathescape=true,escapeinside={(*@}{@*)}]
fn $add$($r_1$, $r_2$)
    $r_3$ $\leftarrow$ $r_1 + r_2$ $\textit{// <-- we are here}$
    if $r_3$   $\rightarrow$ return $r_3$
    else  $\rightarrow$ return $r_3$
\end{lstlisting}

If an instruction is invalid to execute, the execution of the program immediately fails.

\begin{lstlisting}[mathescape=true,escapeinside={(*@}{@*)}]
fn $add$($r_1$, $r_2$)
    $r_3$ $\leftarrow$ $r_1 + r_2$
    $\textit{// v-- this is invalid, so we stop}$
    if $r_3$   $\rightarrow$ return $r_3$
    else  $\rightarrow$ return $r_3$
\end{lstlisting}

\subsection{Terminating Instructions}

A terminating instruction is one that diverges control flow. There are three terminating
instructions: \code{return}, \code{jump}, and \code{if}.

\subsubsection{Return}

\begin{lstlisting}[caption=Return example,mathescape=true,escapeinside={(*@}{@*)}]
fn $one$()
    $r_1$ $\leftarrow$ 1
    $r_2$ $\leftarrow$ $identity$($r_1$)
    return $r_2$

fn $identity$($r_1$)
    return $r_1$
\end{lstlisting}

When a function in JSSAT is called via \code{$f$($x$)}, it
will continue to execute until a \code{return} terminating instruction occurs.
Then, if the return terminating instruction is coupled with a register (i.e. \code{return $r_1$}),
the value inside of the register coupled with the return terminating instruction
will be used as the value of the called function.

\subsubsection{Jump}

The \code{jump} terminating function will continue execution at the function specified,
passing the values in registers specified within the instruction as parameters to the functions

\begin{lstlisting}[mathescape=true,escapeinside={(*@}{@*)}]
fn $main$()
    $r_1$ $\leftarrow$ 1
    $r_2$ $\leftarrow$ $f$($r_1$)
    return $r_2$

fn $f$($r_1$)
    $r_2$ $\leftarrow$ $r_1 + 1$
    jump $g$($r_2$)

fn $g$($r_1$)
    return $r_1$
\end{lstlisting}

It is invalid to \code{jump} to a function without the same number of registers.

\begin{lstlisting}[caption=An invalid jump,mathescape=true,escapeinside={(*@}{@*)}]
fn $f$()
    jump $g$()

fn $g$($r_1$)
    return $r_1$
\end{lstlisting}

Conceptually, a \code{jump} can be thought of as a call and return. In the following
example, $f_1$ and $f_2$ are conceptually similar and result in the same end goal:

\begin{lstlisting}[mathescape=true,escapeinside={(*@}{@*)}]
fn $f_1$()
    jump $one$()

fn $f_2$()
    $r_1$ $\leftarrow$ $one$()
    return $r_1$

fn $one$()
    $r_1$ $\leftarrow$ 1
    return $r_1$
\end{lstlisting}

\subsubsection{If}

The \code{if} terminating instruction will conditionally jump to one of two branches,
determined by whether or not a register is \code{true} or \code{false}.

\begin{lstlisting}[mathescape=true,escapeinside={(*@}{@*)}]
fn $f$($r_1$)
    if $r_1$   $\rightarrow$ jump $a$()
    else  $\rightarrow$ jump $b$()
\end{lstlisting}

\subsection{Linear Instructions}

This section outlines the variety of instructions that JSSAT IR has. We find ourselves
in a peculiar position when designing instructions. When executing JSSAT IR, the
more instructions there are, the slower interpretation and abstract interpretation
will be as there is more work and more information to retain during execution. Thus,
it would be beneficial for execution speed to prioritize a small amount of instructions
with overloaded meaning. However, we want to avoid overloading instructions with
excess meaning otherwise it will be harder for users to get the exact semantics
of their program correct (for example, meaning concatenation of strings but accidentally
adding integers instead).

\subsubsection{Record Instructions}

\paragraph{\code{record.new}} Produces a new record on the heap

\begin{enumerate}
  \item $r_r$ register the record is assigned to
  \item $h$ position in the heap the record bound to $r_r$ is at
\end{enumerate}

\begin{lstlisting}[mathescape=true,escapeinside={(*@}{@*)}]
$r_r$ $\leftarrow$ record.new
\end{lstlisting}

\begin{center}
  Heap \\
  \begin{tabular}{ |c|c| }
    \hline
    \dots & $h$    \\
    \hline
    \dots & $\{\}$ \\
    \hline
  \end{tabular}
\end{center}

\paragraph{\code{record.set}} Creates an association from value to value within a record

\begin{enumerate}
  \item $r_r$ register the record is bound to
  \item $h$ position in the heap the record bound to $r_r$ is at
  \item $r_k$ register the value to be used as a key is bound to
  \item $r_v$ register the value to be used as a value is bound to
  \item $V_k$ value of the key, bound to $r_k$
  \item $V_v$ value of the value, bound to $r_v$
\end{enumerate}

\begin{lstlisting}[mathescape=true,escapeinside={(*@}{@*)}]
record.set $r_r$, $r_k \mapsto r_v$
\end{lstlisting}

\begin{center}
  Heap \\
  \begin{tabular}{ |c|c|c| }
    \hline
    \dots & $h$                            & \dots \\
    \hline
    \dots & $\{ \dots, V_k \mapsto V_v \}$ & \dots \\
    \hline
  \end{tabular}
\end{center}

\paragraph{\code{record.get}} Gets the value at a given key of a record

\begin{enumerate}
  \item $r_r$ register the record is bound to
  \item $h$ position in the heap the record bound to $r_r$ is at
  \item $r_k$ register the value to be used as a key is bound to
  \item $V_k$ value of the key, bound to $r_k$
  \item $V_v$ value of the value
  \item $r_y$ register the result is bound to
\end{enumerate}

\begin{lstlisting}[mathescape=true,escapeinside={(*@}{@*)}]
$r_y$ $\leftarrow$ record.get $r_r$[$r_k$]
\end{lstlisting}

In the success case, $V_v$ is bound to $r_y$:

\begin{center}
  Heap$_{success}$ \\
  \begin{tabular}{ |c|c|c| }
    \hline
    \dots & $h$                            & \dots \\
    \hline
    \dots & $\{ \dots, V_k \mapsto V_v \}$ & \dots \\
    \hline
  \end{tabular}
\end{center}

In the failure case, the instruction is invalid to execute.

\begin{center}
  Heap$_{failure}$ \\
  \begin{tabular}{ |c|c|c| }
    \hline
    \dots & $h$           & \dots \\
    \hline
    \dots & $\{ \dots \}$ & \dots \\
    \hline
  \end{tabular}
\end{center}

\section{Divergence from Implementation}

This specification diverges from the real implementation of JSSAT IR in the JSSAT
compiler. The things not implemented are as follows:

\begin{enumerate}
  \item Tuples
  \item Plethora of number types
  \item Optimization passes for automatic number coercion
\end{enumerate}

\bibliographystyle{plain}
\bibliography{proofpapers.bib}

\end{document}
